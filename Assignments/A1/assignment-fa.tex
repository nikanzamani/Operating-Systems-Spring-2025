\documentclass[12pt]{article}
% \usepackage[utf8]{inputenc}
\usepackage{graphicx}
\usepackage{enumitem}
\usepackage{xcolor}
\usepackage{geometry}
\usepackage{amsmath}
\usepackage{amssymb}
\usepackage{eso-pic}
\usepackage{tikz}
\usepackage{afterpage}
\usepackage[absolute]{textpos}
\usepackage{fancyhdr}
\usepackage{lastpage}
\usepackage{listings}
\usepackage{tcolorbox}
% \usepackage{transparent}

% \usepackage{background}

% \backgroundsetup{
%    scale=1,
%    angle=0,
%    opacity=1, % Set to 1 (let TikZ handle opacity)
%    contents={\includegraphics[width=0.5\textwidth]{images/uni_logo.png}},
%    tikzopacity=0.05, % This is the key for XeLaTeX
%    pages=all,
%    color=black,
%    position=current page.center,
% }

 % For adding backgrounds
% \usepackage{tikz}    % For transparency control

\AddToShipoutPictureBG{%
   \begin{tikzpicture}[remember picture, overlay]
      \node[opacity=0.05] at (current page.center) {%
         \includegraphics[width=0.5\textwidth]{images/uni_logo.png}%
      };
   \end{tikzpicture}%
}

% Layout settings
\geometry{a4paper, margin=1in}
\setlength{\TPHorizModule}{1cm}
\setlength{\TPVertModule}{1cm}

% Header and footer
\pagestyle{fancy}
\fancyhf{}
\lhead{\footnotesize \textbf{\courseName}}
\rhead{\footnotesize \textbf{\assignmentName}}
\renewcommand{\headrulewidth}{0.4pt}
\cfoot{\footnotesize \thepage~/~\pageref{LastPage}}
\renewcommand{\footrulewidth}{0.4pt}

% Custom commands
\newcommand{\courseName}{سیستم‌عامل}
\newcommand{\assignmentName}{تمرین اول}
\definecolor{answerbox}{RGB}{200, 200, 200}

% \newtcolorbox{answerbox}{
%     colback=answerbox,
%     colframe=black,
%     boxrule=0pt,
%     leftrule=2pt,
%     arc=0pt,
%     boxsep=0pt,
%     left=6pt,
%     right=6pt,
%     top=6pt,
%     bottom=6pt,
%     width=\linewidth
% }

% Code listing settings
\lstset{
    basicstyle=\ttfamily\small,
    breaklines=true,
    frame=single,
    numbers=left,
    numberstyle=\tiny\color{gray},
    keywordstyle=\color{blue},
    commentstyle=\color{green!50!black},
    stringstyle=\color{red},
    showstringspaces=false,
    tabsize=4
}

% Persian font settings
\usepackage{xepersian}
\settextfont[
    Path = ./fonts/,
    Extension = .ttf,
]{XB Niloofar} % Name of the font file (without .ttf)
     % Persian main font
\setlatintextfont[
    Path = ./fonts/,
    Extension = .otf,
]{Roboto}

% add a background

\begin{document}
% \thispagestyle{empty}

\begin{titlepage}
    \centering
    \vspace*{\fill}
    
    \includegraphics[width=0.25\textwidth]{images/uni_logo.png}\par
    \vspace{0.3cm}
    
    \rule{0.7\textwidth}{0.4pt}\par
    \vspace{0.5cm}
    
    {\fontsize{14}{16}\selectfont \scshape \courseName}\par
    \vspace{0.1cm}
    {\fontsize{12}{14}\selectfont \textup{دکتر لیلی فرزین‌وش}}\par
    \vspace{0.8cm}
    
    {\fontsize{22}{24}\selectfont \bfseries{\assignmentName}}\par
    \vspace{1cm}
    
    \begin{minipage}{0.5\textwidth}
        \centering
        \begin{tabular}{r@{\hspace{0.5em}}l}
            \textbf{مهلت تحویل:}&  ۳۰ فروردین ۱۴۰۴  \\
            \textbf{ایمیل:}&  zamaninikan@gmail.com \\
        \end{tabular}
    \end{minipage}\par
    
    \vspace{0.5cm}
    \rule{0.4\textwidth}{0.4pt}\par
    \vspace*{\fill}
    
    \begin{minipage}{\textwidth}
        \centering
        \footnotesize\itshape
        *توجه: تمرین را تا ساعت ۱۱:۵۹ شب تاریخ مقرر از طریق کوئرا ارسال کنید. پاسخ‌ها باید تایپ‌شده باشند.
    \end{minipage}
\end{titlepage}

\section{سوالات}
\begin{enumerate}[rightmargin=1cm,leftmargin=2cm]

\item \textbf{تحلیل \lr{Scheduling} اولویت پیشگیرانه (\lr{Preemptive Priority})}

یک سیستم \lr{scheduling} با اولویت پیشگیرانه در نظر بگیرید که:
\begin{itemize}
    % \item 
    \item {پروسه ها با اولویت صفر شروع می‌کنند (اعداد بالاتر = اولویت بالاتر)}
    \item \lr{Process} در حال انتظار (\lr{ready queue}) با نرخ $\alpha$ اولویتشان تغییر می‌کند
    \item \lr{Process} در حال اجرا (\lr{running}) با نرخ $\beta$ اولویتشان تغییر می‌کند
\end{itemize}

این موارد را تحلیل کنید و معادل الگوریتمی آنها را مشخص نمایید:
\begin{enumerate}[label=(\harfi*)]
    \item وقتی $\beta > \alpha > 0$ باشد، چه روش \lr{scheduling} به وجود می‌آید؟
    \item وقتی $\alpha < \beta < 0$ باشد، چه روش \lr{scheduling} به وجود می‌آید؟
\end{enumerate}

\item \textbf{روابط بین الگوریتم‌های \lr{CPU Scheduling}}

الگوریتم‌های \lr{CPU Scheduling} اغلب به صورت گروه های پارامتری وجود دارند. مانند:
\begin{itemize}
    \item \lr{RR (Round Robin)} نیازمند پارامتر \lr{time quantum} است
    \item صف‌های چندسطحی \lr{feedback} نیازمند پارامترهایی برای:
    \begin{itemize}
        \item تعداد صف‌ها
        \item روش‌های \lr{scheduling} هر صف
        \item قوانین جابجایی \lr{process} بین صف‌ها
    \end{itemize}
\end{itemize}

این مجموعه‌الگوریتم‌ها ممکن است همپوشانی داشته باشند (مثلاً \lr{FCFS} = \lr{RR} با \lr{quantum} بی‌نهایت). روابط بین موارد زیر را در صورت وجود بررسی کنید:
\begin{enumerate}[label=(\harfi*)]
    \item \lr{Priority} در مقابل \lr{SJN}
    \item صف‌های چندسطحی \lr{feedback} در مقابل \lr{FCFS}
    \item \lr{Priority} در مقابل \lr{FCFS}
    \item \lr{RR} در مقابل \lr{SJN}
\end{enumerate}

\item \textbf{تحلیل خروجی برنامه \lr{C}}

خروجی کامل این برنامه \lr{C} را مشخص کنید:
\newpage

\begin{latin}
    
\begin{lstlisting}[language=C, basicstyle=\ttfamily\footnotesize, numbers=left]
#include <sys/types.h>
#include <stdio.h>
#include <unistd.h>

int value = 11;

int main() {
    pid_t pid;
    pid = fork();
    value += 5;
    
    if (pid == 0) { /* child */
        printf("%d", value);
        value += 9;
        return 0;
    }
    else { /* parent */
        value += 2;
        wait(NULL);
        printf("%d", value);
        return 0;
    }
}
\end{lstlisting}
\end{latin}

\item \textbf{طراحی جایگزین برای معماری لایه‌ای \lr{OS}}

در طراحی استاندارد لایه‌ای \lr{OS}، سلسله مراتب سختگیرانه وجود دارد (هر لایه فقط از لایه پایینی مجاور استفاده می‌کند). یک روش جایگزین پیشنهاد کنید که در آن عملیات حیاتی از نظر \lr{performance} بتوانند لایه‌ها را دور بزنند، سپس \lr{trade-off}های این روش را بحث کنید.

\item \textbf{محاسبه \lr{CPU Utilization}}

با توجه به موارد زیر:
\begin{itemize}
    \item میانگین زمان \lr{burst} \lr{CPU}: \lr{6ms}
    \item زمان \lr{context switch}: \lr{0.5ms}
    \item \lr{quantum} در \lr{Round Robin}: \lr{4ms}
\end{itemize}

حداکثر \lr{utilization} ممکن \lr{CPU} را در این شرایط محاسبه کنید.

\item \textbf{تحلیل ایجاد \lr{Process}}

برای قطعه کد زیر:
\begin{latin}
\begin{lstlisting}[language=C, basicstyle=\ttfamily\footnotesize, numbers=left]
for (int i = 0; i < 3; i++) {
    fork();
    if (i % 2 == 0) {
        fork();
    }
}
\end{lstlisting}
    
\end{latin}

تعداد کل \lr{process} ایجاد شده (شامل \lr{process} اصلی) را مشخص کنید و پاسخ خود را با رسم درخت \lr{process}ها نشان دهید.

\item \textbf{تحلیل \lr{Performance} چندنخی (\lr{Multithreading})}

مثال هایی برای موارد زیر ارائه دهید:
\begin{enumerate}[label=\arabic*.]
    \item دو مورد مسئله‌ی برنامه‌نویسی که در آنها \lr{multithreading} باعث بهبود \lr{performance} نسبت به حالت تک‌نخی می‌شود
    \item دو مورد مسئله‌ی برنامه نویسی که در آنها \lr{multithreading} باعث کاهش \lr{performance} نسبت به حالت تک‌نخی می‌شود
\end{enumerate}

\end{enumerate}

\begin{tikzpicture}[remember picture,overlay]
\node[anchor=south west, xshift=30mm, yshift=30mm] 
    at (current page.south west) 
    {\footnotesize\itshape باشید موفق };
\end{tikzpicture}

\end{document}
